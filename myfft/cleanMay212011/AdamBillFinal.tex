\documentclass[numreferences]{kluwer}    % Specifies the document style.
\usepackage{graphicx}
\usepackage{setspace}
\newdisplay{guess}{Conjecture}
\newtheorem{Thm}{Theorem}[section]
\newtheorem{Prop}[Thm]{Proposition}
\newtheorem{Lem}[Thm]{Lemma}
\newtheorem{Cor}[Thm]{Corollary}
\newtheorem{Alg}[Thm]{Algorithm}
\newtheorem{Que}[Thm]{Question}
\newtheorem{Conj}[Thm]{Conjecture}
\newtheorem{Rem}{Remark}
\newtheorem{Exa}{Example}
\newtheorem{Con}{Condition}
\newtheorem{Def}[Thm]{Definition}


\begin{document}
\begin{article}
\begin{opening}
\date{April 15,2009}
\title{Pricing American options on exponential Levy processes}

\author{Adam \surname{Kolkiewicz}\email{wakolkie@math.uwaterloo.ca}}
\institute{Department of Statistics and Actuarial Science,
University of Waterloo\\
Waterloo, Ontario, N2L 3G1, Canada}
\author{Xiao Rong \surname{Zou}\email{xrzou@math.uwaterloo.ca}}
\institute{Department of Statistics and Actuarial Science,
University of Waterloo\\
Waterloo, Ontario, N2L 3G1, Canada}

\date{August, 10, 2010}
\runningauthor{A. Kolkiewicz \& X. Zou} \runningtitle{Pricing
American options on exponential Levy processes}
\begin{abstract}
to be added
\end{abstract}
\end{opening}
\section{Introduction and Motivation}\label{intro}
To be added.
\section{the formulation of the algorithm and implementation}
We are assuming that the dynamics of the prices of the underlying
risky security $\{S_t\}_{0\le t\le T}$ follows a process of the form
\[
S_t=e^{X_t}
\]
where $X_t$ follows a Levy process with $X_0=\ln S_0$.


Let us denote the characteristic function of $X_t$ by $\phi_t(v)=
E[e^{iv\cdot X_t}], x\in R$.

We shall focus on the Bermuda put option with $M$ periods. The the
value for call option can be derived by the parity equation. The
algorithm can be used for the other popular options where exercise
payoff has a simple expression in term of underlying asset.

The exercise value at any time $t$ before maturity is
\[
G(S_t)=\left\{\begin{array}{cc}
         (\alpha K-\beta S_t)^+ & S_t\le K \\
         0 & S_t> K
       \end{array}\right.
\]

where $K$ is the strike price and $\alpha,\beta$ are non negative
parameters. Let $\Delta=T/M$, $k=\ln K$. Let
\[
\phi(v)=E[e^{ivX_{\Delta}}].
\]
be the characteristic function. We are going to write $S_{j\Delta}$
and $X_{j\Delta}$ as $S_j$ and $X_j$ for convenience. So
\[
S_{j+1}=e^{X_{j+1}}=e^{X_j}e^{X_{j+1}-X_j} \sim S_je^{Z}
\]
where $Z=X_{j+1}-X_j$ has the density $q(z)$ (without depending on
$j$). We shall scale the $X_j$ by
\[
X_j=\sigma Y_j+\mu
\]
where
\[
\sigma = var(X_j),\quad \mu=E[X_j]
\]
and
\[
Z=X_{j+1}-X_j=\sigma(Y_{j+1}-Y_j)=\sigma W
\]
where the distribution of $W=Y_{j+1}-Y_j$ is independent on $j$.

Define
\[
C^{s}_j(y)=C_j(e^{\sigma y+\mu}),\quad V^{s}_{j}(y)=V_j(e^{\sigma
y+\mu}), \quad G^s(y)=G(e^{\sigma y+\mu})
\]

Let $f_W(w)$ be the density function of $W$ and $f_Z(z)$ be the
density of $Z$, we have
\[
f_W(w)=\sigma f_Z(\sigma w)
\]
so
\begin{eqnarray*}
e^{r\delta} C^s_j(y)&=&e^{r\delta} C_j(e^{\sigma
y+\mu})=E[V_{j+1}(e^{\sigma y+\mu}e^{\sigma W})]\\
&=&\int^{\infty}_{-\infty}V_{j+1}(e^{\sigma y+\mu +\sigma
w})f_W(w)dw\\
&=&\int^{\infty}_{-\infty}V^s_{j+1}(y+w)f_W(w)dw
\end{eqnarray*}
Assume that
\[
f_W(w)|_{[L,R]}\approx F_0/2+\sum_{k=1}^{N-1} F_k
cos(k\pi\frac{x-L}{R-L})
\]
where
\begin{eqnarray*}
F_k&=&\frac{2}{R-L}Re(\int^R_Lf_W(w)e^{i k \pi
(w-L)/(R-L)}dw)\\
&\approx& \frac{2}{R-L}Re( \phi_W(\frac{k\pi}{R-L}) e^{ \pi
kL/(R-L)})
\end{eqnarray*}
where
\[
\phi_W(t)=\int^{\infty}_{-\infty}e^{itw}f_W(w)dw
\]
Let
\[
b_0=-\infty<b_1<\dots b_{k}<cdots<b_{M+1}=R<\infty
\]
be a partition of $(-\infty,\infty)$ and the $V_{j+1}$ is equal to
the payoff at
\[
V^s_{j+1}(y)=G^s(y), \quad -\infty<y\le b_1
\]
and  $V_{j+1}$ can be approximated by $d$-degree polynomial on each
interval $[b_k,b_{k+1})$ for  $(1\le k\le M)$:
\[
V^s_{j+1}(y)=\sum_{h=0}^d c_{k,h}(y-b_k)^{h}, \quad  b_{k}\le y <
b_{k+1},
\]
so
\begin{eqnarray*}
e^{r\delta}C^s_j(y)&=&\int^{b_1-y}_{-\infty}G^s(w+y)f_W(w)dw\\
&+&\sum_{k=1}^{M}
\int^{b_{k+1}-y}_{b_k-y}V^s_{j+1}(y+w)f_W(w)dw\\
&+&\int^{\infty}_{b_{M+1}-y}V^s_{j+1}(w+y)f_W(w)dw\\
 &:=&I+II+III
\end{eqnarray*}
To estimate $I$,let
\[
y_1=\min(L,b_1-y),\quad y_2=\min(\max(L,b-y_1),R), y_3=\max(R,b_1-y)
\]
we discuss $I$ in three cases. We assume that $G^s(y)$ is an
decreasing function,i.e. a put-style option.
\begin{enumerate}
\item $b_1-y\le L$. $y_1=b_1-y$ and $y_2=L$. It is clear that we have
\begin{equation}\label{gsb}
0\le I-\int^{y_2}_L G^s(y+w)f_W(w)dw  \le G^s(-\infty)F_W(L)
\end{equation}
\item $L\le b_1-y\le R$, $y_1=L$ and $y_2=b_1-y$,
\[
I=\int^{y_2}_L G^s(y+w)f_W(w)dw + \int^{L}_{-\infty}
G^s(y+w)f_W(w)dw
\]
It is clear
\[
0\le \int^{L}_{-\infty} G^s(y+w)f_W(w)dw\le G^s(-\infty) F_{W}(L)
\]
so the equation (\ref{gsb}) holds as in the case 1.
\item $b_1-y> R$, $y_1=L$ and $y_2=R$, and
\[
I=(\int^{L}_{-\infty}+\int^{y_2}_L +\int^{b_1-y}_{R})
G^s(y+w)f_W(w)dw
\]
It is straightforward to show
\[
0\le \int^{b_1-y}_{R} G^s(y+w)f_W(w)dw\le G^s(-\infty)(1-F_W(R))
\]
\end{enumerate}
So we have
\begin{equation}
0\le I-\int^{y_2}_{L}G^s(y+w)f_W(w)dw\le G^s(-\infty) (1-P(L\le W\le
R))
\end{equation}
and
\[
0\le III\le V^s_{j+1}(R)P(W\ge R-y)
\]
Let
\[
low(y,k)=\min(\max(b_k-y,L),R),\quad
up(y,k)=\min(\max(b_{k+1}-y,L),R),
\]
then
\begin{eqnarray*}
0&\le& \int^{b_{k+1}-y}_{b_k-y}V^s_{j+1}(y+w)f_W(w)dw-
\int^{up(y,k)}_{low(y,k)}V^s_{j+1}(y+w)f_W(w)dw\\
&\le& V_{j+1}^s(b_k)(1-P(L\le W\le R))
\end{eqnarray*}
\bibliographystyle{amsplain}
\begin{thebibliography}{10}
\bibitem{cama}
P. Carr  and D. Madan,  \textit{Option valuation using the fast
Fourier transform} J. Comput. Finance (2), pp61--73, 1998

\bibitem{como}
P. Costantini and R. Morandi, \textit{Monotone and convex cubic
spline interpolation}, Calcolo, Vol 21, pp 281--294, 1984

\bibitem{losc}
F.A. Longstaff and E.S. Schwartz, \textit{Valuing American options
by simulation: a simple least-squares approach,} Review of Financial
Studies (14), pp 113--147. 2001.

\bibitem{adam}
A. Kolkiewicz, \textit{Pricing American options on exponential Levy
processes,} Dec, 2007.
\end{thebibliography}


\end{article}

\end{document}


%%%%
some diagrams that might be helpful


\begin{figure} [ht]
\centering
\includegraphics{Gauss.eps}
\caption{My figure} \label{the-label-for-cross-referencing}
\end{figure}

\begin{figure} [ht]
\centering
\includegraphics{Merton.eps}
\caption{My figure} \label{the-label-for-cross-referencing}
\end{figure}


\begin{figure} [ht]
\centering
\includegraphics{Kou.eps}
\caption{My figure} \label{the-label-for-cross-referencing}
\end{figure}


\begin{figure} [ht]
\centering
\includegraphics{NIG.eps}
\caption{My figure} \label{the-label-for-cross-referencing}
\end{figure}


\begin{figure} [ht]
\centering
\includegraphics{VG.eps}
\caption{My figure} \label{the-label-for-cross-referencing}
\end{figure}

\begin{figure} [ht]
\centering
\includegraphics{TS.eps}
\caption{My figure} \label{the-label-for-cross-referencing}
\end{figure}


\section{boundaries}

\begin{figure} [ht]
\centering
\includegraphics{Gauss_boundary.eps}
\caption{My figure} \label{the-label-for-cross-referencing}
\end{figure}

\begin{figure} [ht]
\centering
\includegraphics{Merton_boundary.eps}
\caption{My figure} \label{the-label-for-cross-referencing}
\end{figure}


\begin{figure} [ht]
\centering
\includegraphics{Kou_boundary.eps}
\caption{My figure} \label{the-label-for-cross-referencing}
\end{figure}


\begin{figure} [ht]
\centering
\includegraphics{NIG_boundary.eps}
\caption{My figure} \label{the-label-for-cross-referencing}
\end{figure}


\begin{figure} [ht]
\centering
\includegraphics{VG_boundary.eps}
\caption{My figure} \label{the-label-for-cross-referencing}
\end{figure}

\begin{figure} [ht]
\centering
\includegraphics{TS_boundary.eps}
\caption{My figure} \label{the-label-for-cross-referencing}
\end{figure}




\begin{pf}
Let $y_i=f(x_i)$ $(0\le i\le n)$. Define
\[
\triangle_i=\frac{y_{i+1}-y_i}{h},\quad
d_i=\frac{y_{i+1}-y_{i-1}}{2h},\quad 1\le i\le n-1
\]
We define
\[
s(x)=\frac{y_{i-1}-2y_i+y_{i+1}}{2h^2}
(x-x_i)^2+d_i(x-x_i)+y_i,\quad x_{i-1} \le x\le x_i, 1\le i\le n-1
\]
and $s(x)$ over $[t_{n-1},t_n]$ is defined same as over
$[t_{n-2},t_{n-1}]$, i.e.
\[
s(x)=\frac{y_{n-2}-2y_{n-1}+y_n}{2h^2}(x-x_{n-1})^2+d_{n-1}(x-x_{n-1})+y_{n-1},\quad
x_{n-1}\le x \le x_n
\]
It is easy to check
\[
s(x_{i_1})=y_{i-1},\quad s(x_i)=y_i
\]
and
\begin{eqnarray*}
s''(x)&=&\frac{y_{i-1}-2y_i+y_{i+1}}{2h^2}\ge 0, x\in (x_{i-1},x_i),
1\le i\le n-1\\
s''(x)&=&\frac{y_{n-2}-2y_{n-1}+y_n}{2h^2} \ge 0, x\in (x_{n-1},x_n)
\end{eqnarray*}
since $f$ is convex. It is easy to see that,
\begin{eqnarray*}
s'(x_{i-1}+)&=& -\frac{3y_{i-1}-4y_{i}+y_{i+1}}{2h}, \\
s'(x_i-)&=&\frac{y_{i+1}-y_{i-1}}{2h}\\
s'(x_i+)&=& -\frac{3y_{i}-4y_{i+1}+y_{i+2}}{2h},
\end{eqnarray*}
so
\begin{eqnarray*}
s'(x_i+)-s'(x_i-)&=&\frac{y_{i-1}-3y_i+3y_{i+1}-y_{i+2}}{2h}\\
&=&\frac{(y_{i-1}-3y_i+2y_{i+1})+(y_{i+1}-y_{i+2})}{2h}>0
\end{eqnarray*}
since
%\begin{eqnarray*}
%s(x_i)=y_i,\quad s'(x_i^_)=d_i
%\end{eqnarray*}
\end{pf}


\section{Appendix: sydecode code of the implementation}

\begin{figure} [ht]
\centering
\includegraphics{Gauss.eps}
\caption{My figure} \label{the-label-for-cross-referencing}
\end{figure}

\begin{figure} [ht]
\centering
\includegraphics{Merton.eps}
\caption{My figure} \label{the-label-for-cross-referencing}
\end{figure}


\begin{figure} [ht]
\centering
\includegraphics{Kou.eps}
\caption{My figure} \label{the-label-for-cross-referencing}
\end{figure}


\begin{figure} [ht]
\centering
\includegraphics{NIG.eps}
\caption{My figure} \label{the-label-for-cross-referencing}
\end{figure}


\begin{figure} [ht]
\centering
\includegraphics{VG.eps}
\caption{My figure} \label{the-label-for-cross-referencing}
\end{figure}

\begin{figure} [ht]
\centering
\includegraphics{TS.eps}
\caption{My figure} \label{the-label-for-cross-referencing}
\end{figure}
